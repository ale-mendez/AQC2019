\documentclass[10pt]{article}
\usepackage{amsmath,amssymb}
\usepackage{graphicx}
\usepackage{authblk}
\usepackage{float}
\usepackage{xcolor}
\usepackage[margin=1.1in]{geometry}
\usepackage{hyperref}


\hyphenation{pseudo-charges pseudo-potential mo-men-tum lith-ium}
%%%%%%%%%%%%%%%%%%%%%%%%%%%%%%%%%%%%%%%%%%%%%%%%%%%%%%%%%%%%%%%%%%%%%
%%%%%%%%%%%%%%%%%%%%%%%%%%%%%%%%%%%%%%%%%%%%%%%%%%%%%%%%%%%%%%%%%%%%%
\begin{document}

\title{Collision processes in atoms and molecules using effective 
potentials}
\author[ ]{Alejandra M.P. Mendez}
\author[ ]{Dar\'io M. Mitnik}
\author[ ]{Jorge E. Miraglia}
\affil[ ]{\small
Instituto de Astronom\'ia y F\'isica del Espacio,  
Universidad de Buenos Aires -- 
Consejo Nacional de Investigaciones Cient\'ificas y T\'ecnicas,
Buenos Aires, Argentina}

\maketitle

\tableofcontents

\begin{abstract}
We investigate the feasibility of using pseudopotentials to generate
the bound and continuum orbitals needed in collisional calculations. 
By examination of several inelastic processes in the first Born 
approximation, we demonstrate the inconveniences of this approach. 
Instead, we prescribe the usage of effective potentials obtained with 
the depurated inversion method (DIM), which is described in this work. 
We also extended this method for molecular systems. Calculations of 
single first--order photoionisation and proton--impact ionisation with 
DIM show good agreements with experimental results.
\end{abstract}


%%%%%%%%%%%%%%%%%%%%%%%%%%%%%%%%%%%%%%%%%%%%%%%%%%%%%%%%%%%%%%%%%%%%%
\section{Outline}

Inelastic transition calculations require the representation of 
the bound and continuum states involved in the collisional processes. 
The hypothetical existence of an effective one--electron local 
potential accounting for these states would allow generating more 
directly the orthogonal wavefunctions for the interacting particles.
This approach should include individual $nl$--orbital potentials, a 
feature missing in most of the standard density functional methods. 
The idea of replacing a many--body, nonlocal interaction by an 
effective one--electron equation opens up the possibility of studying
extremely complex systems with high accuracy. 

In this context, one bright idea emerges from the pseudopotential
approximation (PPA), in which all the complexity of the wavefunctions near 
the core --that usually consumes a huge numerical effort-- is avoided.
For instance, density functional theory codes using pseudopotentials, 
such as the {\sc parsec} for example~\cite{parsecwebpage,Chelikowsky1994}, 
permit to use an equally--spaced grid involving a relatively small 
number of points. 
Otherwise, the use of realistic potentials describing the nucleus 
Coulomb potential requires a high density of points concentrated at the 
origin to describe what the pseudopotentials cast aside. 
Thus, if PPA were applicable in the field of collisions theory, 
one would save an enormous amount of computational resources.

Another interesting approach is the depurated inversion method 
(DIM) \cite{Mendez2015,Mendez2016,Mendez2018}, which allows
obtaining accurate, effective potentials by substituting the coupled 
multielectron equations into a Kohn--Sham type equation. 
In the first step, the potential is obtained through inversion of the 
one--electron equation. Next, careful optimisation of the potential is 
carried on, eliminating poles, and imposing the appropriate boundary 
conditions analytically. In that way, the DIM potentials are parametrised 
in simple analytical expressions.

In the present work, we explore the possibility of implementing an 
effective potential approximation in the atomic collision theory to 
describe inelastic processes. Mainly, we examine several collisional 
processes involving 
a single electron transition: photoionisation, excitation, ionisation 
and electron capture. To this end, several simplifications were made: 
(1) The calculation is restrained to Hamiltonians describing only
the moving projectile, the target and the active 
electron. (2) The transition matrix elements are only considered in  
first perturbative order. If the first order description fails, 
indeed it would not have any sense to extend the calculation to 
higher terms of the series.
A wide variety of {\it ab initio} methods have been 
implemented to compute the scattering cross sections for atomic targets, 
from the early implementations of the first Born 
approximation (FBA)~\cite{BatesBook,McDowell1961}, to the much 
more sophisticated time--dependent close--coupling 
methods~\cite{LeePindzola2011,Pindzola2016}. For simplicity, we will 
restrict our calculations only to the FBA framework, 
which is known to give reasonable agreement with 
the experimental cross section in the intermediate--high projectile 
energy range. Moreover, within this energy range and approximation 
order, the Hartree--Fock orbitals are known to provide the correct 
high energy limit.

In Section~\ref{sec:colproc}, we examine the above mentioned 
inelastic processes for two atoms with a single outer electron: 
hydrogen and lithium. 
In this context, we inspect the influence of the target description 
in the cross sections when the PPA and DIM approaches are implemented. 
Furthermore, these effects have been previously studied in other 
perturbative approaches, i.e. the continuum distorted wave 
eikonal--initial--state (CDW-EIS), for various targets (for example,
see~\cite{Kirchner1998,Fiori2001}).
Finally, our first--order DIM photoionisation cross sections for 
many--electron atoms are compared with experimental measurements. 

On the other hand, the description of molecular systems constitutes 
a real challenge due to their nonspherical symmetry and multicenter 
character. Many \textit{ab initio} and semi--empirical theoretical 
approximations \cite{Szabo1996,Helgaker2000,Schaefer2004} have been 
developed to this end over the last century. 
In this work, we present an extension of the DIM method for simple 
molecular systems, providing a new parametric expression for the 
potentials. 
The target description is once again tested by examination of its 
performance in first--order collisional processes. 
An implementation example is given for the methane molecule.


%%%%%%%%%%%%%%%%%%%%%%%%%%%%%%%%%%%%%%%%%%%%%%%%%%%%%%%%%%%%%%%%%%%%%
\section{Theory}

\subsection{Pseudopotential Approximation}
\label{sec:PPAs}

The pseudopotential approximation consists of replacing the Coulomb 
potential in the many--electron system Hamiltonian with a smooth 
function so that the electron wavefunctions oscillating rapidly in 
the core region are replaced by nodeless pseudo--orbitals having the 
right energy and the same outer range properties.
In general, the pseudopotentials $V_{\mbox{\tiny PPA}}$ can be defined 
through a pseudo--charge $Z_{\mbox{\tiny PPA}}$ as
\begin{equation}
 V_{\mbox{\tiny PPA}}(r)=-\frac{Z_{\mbox{\tiny PPA}}(r)}{r} \,,
 \label{eq:pseudopotential}
\end{equation}
\begin{equation}
 Z_{\mbox{\tiny PPA}}(r)=\bigg\{
 \begin{array}{cl}
  f(r), & r\leq r_c \\
  1, & r>r_c 
 \end{array} 
 \,,
 \label{eq:pseudocharge}
\end{equation}
where $r_c$ is a cutoff radius that separates the core, $r\leq r_c$,
from the valence region, $r>r_c$, of the target and $f(r)$ is a 
continuous function with a constant value at the origin.
Fig.~\ref{fig:pseudoLi} illustrates a pseudopotential (solid line)
and its corresponding pseudo--wavefunction for the $3s$ orbital of 
argon. Notice that the pseudopotential behaves as $-1/r$ (dot--dash 
line) in the valence region, as defined in Eqs. (\ref{eq:pseudopotential})
and (\ref{eq:pseudocharge}).
The one--electron Hartree--Fock wavefunction is included
with a dashed line. The pseudo--wavefunction agrees with the 
HF orbital only in the outer region, losing all 
information about the atomic structure close to the origin.
\begin{figure}[H]
\centering
 \includegraphics[height=0.23\textheight]{figures/pseudopot/pseudopotAr.eps}
\hspace{0.2cm}
 \includegraphics[height=0.23\textheight]{figures/pseudopot/pseudofuncAr.eps}
 \caption{(a) Pseudopotential, (b) pseudo--wavefunction and 
 HF orbital for the $3s$ orbital of argon.}
 \label{fig:pseudoLi}
\end{figure}

In Section~\ref{sec:colproc}, we analyse the feasibility of 
implementing pseudopotentials in collisional processes calculations for two 
simple atomic targets: hydrogen and lithium. For each atom, the 
following pseudopotentials are examined
\begin{equation}
 \begin{array}{clll}
  \mbox{Name} & \mbox{Source} & \mbox{Type} & \mbox{Ref.} \\
  \hline  
  A & \mbox{\sc abinit} & \mbox{GGA} & \cite{abinitwebpage,Hamann1979} \\
  P & \mbox{\sc parsec} & \mbox{Troullier Martins} & \cite{parsecwebpage,Chelikowsky1994}\,.
 \end{array}
 \label{eq:pseudosources}
\end{equation}
The hydrogen atom has only one electron, and it does not seem to have 
any use to calculate the corresponding pseudopotential. However, the 
hydrogen pseudopotentials from (\ref{eq:pseudosources}) reproduce with
high accuracy the main features of the wavefunctions, even for excited
states.


We will now proceed to examine the pseudo--charges and its 
one--electron solutions for the lithium atom closely. 
First, we study the spatial and momentum representation of the 
pseudo--charges. The momentum--space equivalent of $Z(r)$ is given 
by the Fourier transform
\begin{equation}
 \widetilde{Z}(k) = \frac{1}{\sqrt{2\pi}} \int_{-\infty}^{+\infty} 
 Z(r)\,e^{-ikr}\,dr\,.
\end{equation}
The pseudo--charges from~(\ref{eq:pseudosources}) for the $2s$ 
orbital of lithium are illustrated in Fig.~\ref{fig:ZLi}. 
For comparison, we include the potential attained from implementing 
the depurated inversion method described in Section~\ref{sec:DIM}.
The pseudo--charges vanish at the origin, avoiding the divergence 
of the Coulomb potential. However, this feature comes at a price: 
the pseudo--charges in the spatial representation are repulsive 
around \mbox{$r= $1 a.u.}, and their momentum picture fails to represent the 
target for high $k$, showing an incorrect oscillatory behaviour for 
values greater than $k_c=(2\pi r_c)^{-1}\sim0.7$ a.u..
\begin{figure}[H]
\centering
 \includegraphics[height=0.23\textheight]{figures/pseudopot/ZLi.eps}
\hspace{0.2cm}
 \includegraphics[height=0.23\textheight]{figures/pseudopot/ZLik.eps}
 \caption{Pseudo and DIM charges for the $2s$ orbital of lithium. 
 (a) Spatial and (b) momentum representation.}
 \label{fig:ZLi}
\end{figure}

Secondly, we inspect the behaviour of the bound pseudo--orbitals 
obtained from solving the one--electron Schr\"odinger equation with a 
pseudopotential. As usual, the bound state wavefunctions can be 
written as
\begin{equation}
 \psi_{nlm}(\mathbf{r}) = \frac{u_{nl}(r)}{r}Y_l^m(\hat{r})\,,
 \label{eq:centralfield-wave}
\end{equation}
where $u_{nl}(r)$ are the reduced radial wavefunctions, and 
$Y_l^m(\hat{r})$ are the spherical harmonics. Similarly, the Fourier 
transform of these functions is given by
\begin{equation}
 \widetilde{\psi}_{nlm}(\mathbf{k}) =
 \frac{\chi_{nl}(k)}{k}Y_l^m(\hat{k})\,.
\end{equation}

The spatial and momentum representations of the $2s$ radial 
pseudo--wavefunctions of lithium corresponding to the pseudo--charges 
from~(\ref{eq:pseudosources}) are displayed in Fig.~\ref{fig:wavesLi}.
Although the pseudo--orbitals are very different
from the HF $2s$ wavefunction, the transformed $\chi(k)$ seems to 
have similar characteristics. However, a closer inspection of the 
tail region of these functions (see the inset of the figure) 
shows the existence of several nodes.
We will see later that these discrepancies introduce significant 
consequences in the cross sections for most of the collisional 
processes examined.
\begin{figure}[H]
\centering
 \includegraphics[height=0.23\textheight]{figures/pseudopot/funcLi.eps}
\hspace{0.2cm}
 \includegraphics[height=0.23\textheight]{figures/pseudopot/funckLi.eps}
 \caption{Pseudo and DIM bound state wavefunction in (a) spatial
 and (b) momentum representation for the
 $2s$ orbital of lithium.}
 \label{fig:wavesLi}
\end{figure}

Finally, the pseudopotential approach not only affects the 
representation of the bound orbitals but also determines the form of 
the continuum wavefunctions. For large $r$, the free state orbitals of 
an electron in the presence of a Coulomb potential can be written as
\begin{equation}
 u_{kl}(r) \rightarrow \,\sin \left( kr - l\frac{\pi}{2} - \eta \ln 2kr +
 \sigma_l + \delta_l \right)\,,
\end{equation}
where $k$ is the particle wave number, $\eta$ is Sommerfeld's 
parameter, $\sigma_l$ is the Coulomb phase shift and $\delta_l$ is the 
wave phase shift with respect to the Coulomb wave.

Comparisons between the DIM (solid line) and the pseudo (dashed)
free $ks$ wavefunctions for lithium are shown in Fig.~\ref{fig:contLi},
close to the origin (left) and asymptotically (right).
The pseudo and DIM wavefunctions behave similarly far away from the nucleus.
The asymptotic phase shift $\Delta$ accounts for the differences 
between the potentials. As the 
energy of the free electron increases, $\Delta$ diminishes. 
However, the orbitals in the core region are different even with 
increasing energy; the first maximum of the DIM wavefunctions 
are consistently smaller than of the pseudo--orbitals, which is 
understood since the Coulomb--type attraction of the nuclei is 
stronger than the pseudopotential in that region. 
\begin{figure}[H]
\centering
\includegraphics[height=0.23\textheight]{figures/pseudopot/LicontA.eps}
% \hspace{0.01cm}
\includegraphics[height=0.23\textheight]{figures/pseudopot/LicontB.eps}
 \caption{Continuum wavefunctions with energies $E$ near the origin 
 (left) and in the asymptotic region (right), calculated with the 
 DIM potential (solid line) and the {\sc abinit} pseudopotential (dashed 
 line). }
 \label{fig:contLi}
\end{figure}

%%%%%%%%%%%%%%%%%%%%%%%%%%%%%%%%%%%%%%%%%%%%%%%%%%%%%%%%%%%%%%%%%%%%%
\subsection{Depurated Inversion Method Potentials}
\label{sec:DIM}

The depurated inversion method \cite{Mendez2015,Mendez2016,Mendez2018} 
consists of assuming that the many--electron atom orbitals can be 
represented by the solution of Kohn--Sham type equations, in which 
the $nl$ effective potentials are given by 
\begin{equation}
V_{nl}(r) = 
\frac{1}{2}\frac{1}{u_{nl}(r)}
\frac{d^2u_{nl}(r)}{dr^{2}} - 
\frac{l(l+1)}{2r^{2}}+\varepsilon_{nl} \, ,
\label{eq:V}
\end{equation}
where $u_{nl}$ and $\varepsilon_{nl}$ are the $nl$ orbital wavefunctions and
energies, respectively. In this work, the atomic structure is
approximated with the Hartree--Fock method, 
which is computed with the {\sc hf} codes by 
C. Froese Fischer \cite{FroeseFischer1997} and the {\sc nrhf} code by 
W. Johnson \cite{Johnson2007}. The computation of Eq.~(\ref{eq:V}) 
poses various numerical problems. The nodes and asymptotic decay
of the wavefunctions $u_{nl}(r)$ introduce significant numerical errors in
the inversion procedure (see Ref.~\cite{Mendez2018} for further 
details). The nodes of the orbitals produce huge unphysical 
poles, while the rapid asymptotic decay of the internal wavefunctions 
generates large divergences in the tail region of the potentials.
The Depuration method is implemented to tackle these unphysical features. 
An effective potential with a Coulomb--type shape $V_r(r)=-Z_r(r)/r$ 
is defined, and we enforce the correct boundary 
conditions fitting the inverted potential with the following analytical 
expression
\begin{equation}
 Z_r(r) = \sum_{j=1}^n z_j e^{-\alpha_j r}(1+\beta_jr) + 1
 \, \longrightarrow \, \bigg\{
 \begin{array}{cc}
  Z_N, & r\rightarrow0 \\
  1, & r\rightarrow\infty 
 \end{array}
 \label{eq:atomiczDIM}
\end{equation} 
where \mbox{$\sum z_j = Z_N-1$}. Afterwards, the parameters are 
optimised to reproduce the HF values accurately. 

%%%%%%%%%%%%%%%%%%%%%%%%%%%%%%%%%%%%%%%%%%%%%%%%%%%%%%%%%%%%%%%%%%%%%
\section{Collisional Processes in Atoms}
\label{sec:colproc}

The most significant advantage of the pseudopotential method is 
its simplicity. However, it is worth to determine the validity
of this approach for computing collisional processes.
In this Section, we perform a thorough examination of the
pseudo--potentials for hydrogen and 
lithium by comparing the cross sections of four inelastic 
processes: proton--impact excitation, proton--impact ionisation,
charge exchange and photoionisation. The initial and final states
of the targets are obtained by solving the corresponding Schr\"odinger 
equation.
For the hydrogen atom, we compare the pseudopotential results with 
the exact analytical solutions.  
Furthermore, in order to asses the applicability of the 
depurated inversion method, we compute the photoionisation of 
more complex many--electron atoms and 
compare our findings with experimental data.


%%%%%%%%%%%%%%%%%%%%%%%%%%%%%%%%%%%%%%%%%%%%%%%%%%%%%%%%%%%%%%%%%%%%%
\subsection{Proton--Impact Excitation}

The proton--impact excitation of target $X$ is defined as 
\begin{equation}
 \mbox{H}^+ + X \rightarrow \mbox{H}^+ + X^*\,.
\end{equation}
The excitation cross section $\sigma$ of the target from the initial bound
state $\psi_i$ to the excited state $\psi_{\!f}$ may be written as 
\begin{equation}
 \sigma=\frac{\mu^2}{4\pi^2}\frac{k_{\!f}}{k_i}\int{\left|T_{\!fi}\right|^2 
 d\Omega}\,,
  \label{eq:cross-section}
\end{equation}
where $\mu$ is the reduced mass of the proton--atom system, 
$\mathbf{k}_i$ and $\mathbf{k}_{\!f}$ are the initial and final relative 
momenta, and 
\begin{equation}
 T_{\!fi}=\langle \psi_{\!f}|V|\psi_i\rangle
 \label{eq:transition-matrix}
\end{equation}
is the transition matrix or T--matrix.
If the initial and final states of the transition are described by
the Hartree--Fock method, the orbitals will give the correct high 
energy limit in the first order approximation (this is not the case 
for the charge exchange process). Hence, we will concentrate our 
computing efforts in the first perturbative order of the transition
matrix element through the first Born approximation (FBA), given by
\begin{equation}
 T_{\!fi}^{\mbox{\scriptsize FBA}} = \widetilde{V}(\mathbf{p}) F_{\!fi}(\mathbf{p}) \,.
 \label{eq:firstbornTmatrix}
\end{equation}
The term $F_{\!fi}(\mathbf{p})$ is the form--factor
\begin{equation}
 F_{fi}(\mathbf{p}) = \frac{1}{(2\pi)^{3/2}} 
 \int{\widetilde{\psi}_{\!f}^*(\mathbf{k})
 \widetilde{\psi}_i(\mathbf{k}+\mathbf{p})\,d\mathbf{k}}\,,
 \label{eq:form-factor}
\end{equation}
where $\mathbf{p}$ is the momentum transfer vector
\begin{equation}
 \mathbf{p} = p_{\mbox{\tiny min}} \hat{\mathbf{v}} + \boldsymbol{\eta} \,,
\end{equation}
\begin{equation}
 p_{\mbox{\tiny min}} = \frac{\varepsilon_{\!f}-\varepsilon_i}{v} \rightarrow 
 \bigg\{
 \begin{array}{cl}
  \infty, &v\rightarrow 0 \\
  0, &v\rightarrow \infty
 \end{array}
 \,,
 \label{eq:pmin}
\end{equation}
$\hat{\mathbf{v}}$ is the ion velocity, $\boldsymbol{\eta}$ is
the transversal momentum transfer, so that 
$\hat{\mathbf{v}} \cdot \boldsymbol{\eta}=0$, whereas $\varepsilon_i$ 
and $\varepsilon_{\!f}$ are the binding energies corresponding to the 
initial and final states.

The first Born proton--impact excitation cross sections of hydrogen 
and lithium from the ground states are shown in Fig.~\ref{fig:exHLi}.
The pseudopotential results for the $n=2$ and $n=3$ final states of
hydrogen agree excellently with the analytical expression. For 
lithium, the pseudopotential cross sections agree in a broad velocity 
range with the DIM calculations, except for low proton--impact velocities. 
This disagreement arises from the form 
factor. For low impact velocities, the momentum transfer vector is large 
(\ref{eq:pmin}). As discussed earlier, the bound momentum orbital
$\widetilde{\psi}(\mathbf{k}+\mathbf{p})$ is not described adequately 
by the pseudopotentials at this region. 
An alternative expression for the form factor can
be considered by implementing the peaking approximation
\begin{equation}
 F_{\!fi}(\mathbf{p}) \sim \widetilde{\psi}_i(\mathbf{p})\widetilde{\psi}_{\!f}^*(0)
 +\widetilde{\psi}_{\!f}(\mathbf{p})\widetilde{\psi}_i^*(0)\,.
\end{equation}
Therefore, in order to have the correct form factor, it is necessary to 
obtain an accurate description of the initial bound state at large
momentum values, which is not the case for the pseudostates (see 
Fig.~\ref{fig:wavesLi}b) and hence their failure.

\begin{figure}[H]
\centering
 \includegraphics[height=0.23\textheight]{figures/pseudopot/Hex.eps}
 \hspace{0.4cm}
 \includegraphics[height=0.23\textheight]{figures/pseudopot/Liex.eps}
 \caption{Proton--impact excitation cross section from the ground state 
 for hydrogen and lithium.}
 \label{fig:exHLi}
\end{figure}


%%%%%%%%%%%%%%%%%%%%%%%%%%%%%%%%%%%%%%%%%%%%%%%%%%%%%%%%%%%%%%%%%%%%%
\subsection{Proton--Impact Ionisation}

The transition matrix (\ref{eq:transition-matrix}) for the 
proton--impact ionisation of $X$,
\begin{equation}
 \mbox{H}^+ + X \rightarrow \mbox{H}^+ + X^+ + e^-\,,
\end{equation}
can also be written in terms of the first order Born
approximation. In this case, the final state $\psi_{\!f}$ in 
Eq.~(\ref{eq:form-factor}) is an outgoing continuum 
wavefunction $\psi_{\mathbf{k}_{\!f}}^-$, while 
$\varepsilon_{\!f}=k_{\!f}^2/2$ is the energy of the ionised electron. 

The single--differential proton--impact ionisation cross sections 
$d\sigma/d\varepsilon_{\!f}$ of hydrogen and lithium at a proton 
velocity of $v_p=1$\,a.u. 
are shown in Fig.~\ref{fig:ionHLi}. 
In the case of hydrogen, the pseudopotential and analytical results
agree for all the electron energy range, except at very high values. 
On the other hand, for lithium, the cross sections computed with 
pseudopotentials only agree at low energies. 
Once again, assuming that $\psi_{\mathbf{k}_{\!f}}^-(\mathbf{k})$ can be 
approximated by a plane wave, the form factor is reduced to the 
Fourier transform of the initial bound state
\begin{equation}
 F_{\!fi}(\mathbf{p})\sim\widetilde{\psi}_i(\mathbf{p}-\mathbf{k}_{\!f})\,.
\end{equation}
Then, as $k_{\!f}$ increases, so does $p_{\mbox{\tiny min}}$, and the form
factor is not well represented by the pseudopotentials. The significant 
discrepancies shown in Fig. \ref{fig:ionHLi} provides another demonstration
of how a wrong description of the momentum space wavefunction may
produce huge errors in collisional processes calculations.
\begin{figure}[H]
\centering
 \includegraphics[height=0.225\textheight]{figures/pseudopot/ionizationH.eps}
 \hspace{0.2cm}
 \includegraphics[height=0.225\textheight]{figures/pseudopot/ionizationLi.eps}
 \caption{Single differential proton--impact ionisation cross 
 section for hydrogen and lithium at \mbox{$v_p=1$\, a.u..}}
 \label{fig:ionHLi}
\end{figure}

%%%%%%%%%%%%%%%%%%%%%%%%%%%%%%%%%%%%%%%%%%%%%%%%%%%%%%%%%%%%%%%%%%%%%
\subsection{Proton--Impact Charge Exchange}

The proton--impact charge exchange of target $X$ is defined as
\begin{equation}
 \mbox{H}^+ + X \rightarrow \mbox{H} + X^+\,.
\end{equation}
The charge transfer cross section by the collision of a proton 
(electron capture) is 
computed with the first order Brinkman--Kramers 
approximation~\cite{Brinkman1930}. 
Accordingly, the matrix element is defined by 
\begin{equation}
 T_{\!fi}^{\mbox{\tiny BK}} = \widetilde{\psi}_f^*(\mathbf{W}_{\!f})
 \left[\varepsilon_{\!f}-\frac{W_{\!f}^2}{2} \right]
 \widetilde{\psi}_{\!i}(\mathbf{W}_i)\,,
\end{equation}
where $\mathbf{W}_{\!i}$ and $\mathbf{W}_{\!f}$ are the momentum transfer 
vectors 
\begin{equation}
 \mathbf{W}_{\!i} = W_{\!i0} \hat{\mathbf{v}} + \boldsymbol{\eta}, \quad 
 W_{\!i0} = \frac{v}{2} - p_{\mbox{\tiny min}}
 \label{eq:momtransfi0}
\end{equation}
\begin{equation}
 \mathbf{W}_{\!f} = W_{\!f0} \hat{\mathbf{v}} + \boldsymbol{\eta}, \quad 
 W_{\!f0} = \frac{v}{2} + p_{\mbox{\tiny min}}\,,
 \label{eq:momtransff0}
\end{equation}
and they satisfy the condition $\mathbf{W}_{\!i}+\mathbf{W}_{\!f}=\mathbf{v}$,
and $p_{\mbox{\tiny min}}$ is defined in Eq.~(\ref{eq:pmin}).

The charge exchange cross sections of hydrogen and lithium in the 
ground state are illustrated in Fig.~\ref{fig:captureH}. The cross 
section of hydrogen is described with high accuracy by the 
pseudopotential approach for a wide range of proton velocities. 
However, this process constitutes 
a symmetrical resonance, i.e. $\varepsilon_{\!f}=\varepsilon_{\!i}$, 
and the agreement may be misleading. For the lithium case, the 
pseudopotentials fail utterly to describe the electron capture 
correctly at low and high velocities. For low and high $v_p$ values, the 
momentum transfer vector becomes large, and therefore, the cross 
sections are wrongly calculated with pseudopotentials. 
The results disagree completely for most of the energy values, except
for a minimal range of velocities around $v\simeq0.77$\,a.u..
\begin{figure}[H]
\centering
 \includegraphics[height=0.23\textheight]{figures/pseudopot/captureH.eps}
 \hspace{0.3cm}
 \includegraphics[height=0.23\textheight]{figures/pseudopot/captureLi.eps}
 \caption{Proton--impact electron capture cross section for hydrogen and 
 lithium.}
 \label{fig:captureH}
\end{figure}

%%%%%%%%%%%%%%%%%%%%%%%%%%%%%%%%%%%%%%%%%%%%%%%%%%%%%%%%%%%%%%%%%%%%%
\subsection{Photoionisation}
\label{sec:photoatom}

The single photoionisation is defined as
\begin{eqnarray}
 \hbar\omega + X &\rightarrow& X^+ + e\,.
\end{eqnarray}
Considering a perturbative photon field, the initial bound $\psi_{i}$ 
and final continuum $\psi_{\mathbf{k}_{\!f}}^-$ states of the target are
not considerably distorted; therefore, the relevant matrix element of the 
photoionisation process is given by
\begin{equation}
 T_{\mathbf{k}}^{\mbox{\tiny Ph}} = \int{
 \psi_{\mathbf{k}_{\!f}}^-(\mathbf{r}) 
 \left[-i \hat{\boldsymbol{\varepsilon}}_\lambda \cdot 
 \boldsymbol{\nabla}_\mathbf{r}\right] 
 \psi_{i}(\mathbf{r})}\,,
\end{equation}
where $\hat{\boldsymbol{\varepsilon}}_\lambda$ is the polarisation 
versor and $\mathbf{k}_{\!f}=\sqrt{2(\omega+\varepsilon_i)}$, as imposed 
by energy conservation.

The first--order photoionisation cross sections of hydrogen and 
lithium are shown in Fig.~\ref{fig:photoHLi}. The pseudopotentials 
results for the hydrogen atom agree with the exact analytical 
expression results only for low photon energies, failing at larger values. 
These discrepancies can be understood considering the continuum 
wavefunction $\psi_{\mathbf{k}_{\!f}}^-(\mathbf{r})$ as a plane wave. 
Consequently, the matrix element $T_{\mathbf{k}}^{\mbox{\tiny Ph}}$ 
is reduced to
\begin{equation}
 T_{\mathbf{k}}^{\mbox{\tiny Ph}} \sim 
 -\left(\hat{\boldsymbol{\varepsilon}}_\lambda \cdot \mathbf{k}_{\!f} \right)
 \widetilde{\psi_i} \left(\mathbf{k}_{\!f}\right)\,,
\end{equation}
and it is determined entirely by the behaviour of the bound target 
pseudostate in the momentum representation. For hydrogen, the 
pseudo--orbital from {\sc parsec} in the Fourier space coincides with 
the exact 
analytical solution for the entire range of $k$, which explains the 
excellent agreement in the cross section results.
For lithium, the pseudopotential cross sections disagree with the DIM 
results for all energy values. The large oscillations in the cross sections
are originated by the spurious oscillatory structure of the bound state
for large  $k$ values (see inset of Fig.~\ref{fig:wavesLi}b).

\begin{figure}[H]
\centering
 \includegraphics[height=0.23\textheight]{figures/pseudopot/photoionH.eps}
 \hspace{0.2cm}
 \includegraphics[height=0.23\textheight]{figures/pseudopot/photoionLi.eps}
\caption{Single photoionisation cross section for hydrogen 
and lithium.}
 \label{fig:photoHLi}
\end{figure}

%%%%%%%%%%%%%%%%%%%%%%%%%%%%%%%%%%%%%%%%%%%%%%%%%%%%%%%%%%%%%%%%%%%%%
\subsection{DIM Photoionisation of Many--electron Atoms}

In order to asses the applicability of the depurated inversion method
for atoms with a more complex structure, we compute the 
photoionisation of many--electron targets with the DIM 
potentials~\cite{Mendez2016} and compare our results with 
experimental values. 
The first order photoionisation cross section of nitrogen and neon 
are shown in Fig.~\ref{fig:photoDIM}. Experimental data 
from~\cite{Henke1993,Samson1990,Samson2002,Stolte2016} is illustrated 
with hollow symbols. The DIM photoionisation cross section
of these atoms agree excellently with the experimental values for 
low, medium and high photon--energies. For neon, discrepancies start 
to be noticeable for low and intermediate energy. An accurate 
photoionisation description of heavier atoms requires the inclusion 
of many--body effects that can be relevant, such as orbital 
relaxation due to the creation of a hole, collective response 
of inner shell electrons~\cite{Ederer1964} and correlation effects.

\begin{figure}[H]
\centering
 \includegraphics[height=0.23\textheight]{figures/dimpot/photoN.eps}
 \hspace{0.2cm}
 \includegraphics[height=0.23\textheight]{figures/dimpot/photoNe.eps}
\caption{Single photoionisation cross section for nitrogen and 
neon.}
\label{fig:photoDIM}
\end{figure}


%%%%%%%%%%%%%%%%%%%%%%%%%%%%%%%%%%%%%%%%%%%%%%%%%%%%%%%%%%%%%%%%%%%%%
\section{Depurated Inversion Method for Molecules}

The depurated inversion method described above is extended to determine 
effective potentials for molecules. The theoretical methods are 
established in this Section, and results for methane are given. 
Furthermore, the molecular description of CH$_4$ given by DIM is 
tested by computing two collisional processes within the FBA.

\subsection{Theory}

Without loss of generality, we will present the DIM theoretical grounds 
for hydride compounds. The Hamiltonian of an $N$--electron $X\!H_n$ 
molecule within the Born--Oppenheimer approximation is given by
\begin{equation}
 \mathcal{H}=-\sum_{i=1}^N \frac{1}{2} \nabla^2_{\mathbf{r}_i} 
 - \sum_{i=1}^N \frac{Z_N}{r_i} 
 + \sum_{i=1}^N V_H(r_i)
 + \sum_{i<1}^N \frac{1}{r_{ij}}\,,
\end{equation}
\begin{equation}
 V_H(r_i) = -\sum_{j=1}^{n} \frac{1}{\left|\mathbf{r}_i-\mathbf{R}_{H_j}\right|}\,,
\end{equation}
where $Z_N$ is the nuclear charge of the heavier atom, and 
$\mathbf{R}_{H_j}$ are the coordinates of the hydrogens respect to 
the $X$ atom. The corresponding Schr\"odinger equation 
$\mathcal{H}\Psi=E\Psi$ is solved and the orbitals are expressed as
Eq.~(\ref{eq:centralfield-wave}) considering the single--centre
expansion (SCE). The orbitals and energies are found by solving the 
Hartree--Fock equations. The computation of these 
equations generally relies on the use of finite basis sets for the 
representation of the molecular orbitals (MOs). Usually, the MOs are 
expressed as a linear combination of atomic orbitals (LCAO),
\begin{equation}
 \Psi_i=\sum_j c_{ji} \phi_j\,,
\end{equation}
which can be constructed with Gaussian--type orbitals (GTO) basis 
sets. 

The inverted molecular potential expression, analogous 
to Eq.~(\ref{eq:V}), obtained from GTO basis sets present more difficulties
than the atomic case. In addition to the asymptotic divergences and
the poles, large unphysical 
oscillations arise~\cite{Schipper1997,Mura1997,Jacob2011,Gaiduk2013}. 
These prominent oscillations are originated from imperceptible undulations 
present in the MOs due to the finite number of the basis set. The
second derivative, necessary to evaluate the inversion formula, amplifies
these features~\cite{Schipper1997,Gaiduk2013}.
The appearance of these oscillations in the inverted potentials forces us to
incorporate further actions in the depuration scheme. To illustrate 
this procedure, we consider the $1s$ orbital of the carbon atom. 
We solved the Hartree--Fock equations using the \mbox{6-311G} basis set 
with {\sc gamess} code~\cite{Schmidt1993,Gordon2005}.
We obtain inverted potentials by implementing Eq.~(\ref{eq:V}). 
The resulting $Z_{1s}^{\mbox{\scriptsize 6-311G}}$ charge is shown in 
Fig.~\ref{fig:1sCarbon}a with a dot--dashed line. 
The charge oscillates significantly at low distances and diverges for 
higher $r$ values. The same calculation was 
repeated using the universal Gaussian basis set (UGBS), which has a 
more significant amount of primitives. The corresponding inverted 
charge $Z_{1s}^{\mbox{\scriptsize UGBS}}$ is exhibited in the figure 
with a dashed line. 
Although the charge still diverges around $r\approx1\,$a.u., the 
oscillations are now circumscribed near the nucleus.
Finally, the differential Hartree--Fock equations for the carbon 
atom were solved using the finite--differences (FD) method. 
The $1s$ inverted charge obtained with this procedure, 
$Z_{nl}^{\mbox{\scriptsize FD}}$, shows no oscillations since no basis 
sets have 
been used to construct the orbital; however, the charge still 
diverges for $r>1\,$a.u., as it usually does for all HF calculations.

\begin{figure}[H]
\centering
\includegraphics[height=0.23\textheight]{figures/dimpot/1sCGMSS_BS.eps}
\hspace{0.3cm}
\includegraphics[height=0.23\textheight]{figures/dimpot/1sC_oscprof.eps}
\caption{(a) Effective charges for the $1s$ orbital of carbon. 
(b) Basis--set oscillation profiles.}
\label{fig:1sCarbon}
\end{figure}

As shown in the figure, for each basis set used in the calculations,
different oscillation profiles will arise, which are defined as 
\begin{equation}
 p_{nl}^{\mbox{\scriptsize BS}} = Z_{nl}^{\mbox{\scriptsize BS}}-
 Z_{nl}^{\mbox{\scriptsize FD}} \,,
 \label{eq:oscillation-prof}
\end{equation}
where $Z_{nl}^{\mbox{\scriptsize BS}}$ is the inverted charge of the atom 
using the particular basis set ``BS'' and 
$Z_{nl}^{\mbox{\scriptsize FD}}$ is the effective charge obtained 
from the inversion of the finite--difference wavefunctions. 
In the previous example, the basis set considered for calculating 
the $1s$ orbital of carbon were \mbox{6-311G} and UGBS. The 
oscillation profiles for the $1s$ orbital, using Eq.~(\ref{eq:oscillation-prof}) 
for these basis sets, are shown in Fig.~\ref{fig:1sCarbon}b. 
Since the orbital profiles for each atomic basis set are distinctive,
once they are determined for the atomic case, they can be removed 
in further molecular calculations. An example of this procedure 
is given in the following Section.

%%%%%%%%%%%%%%%%%%%%%%%%%%%%%%%%%%%%%%%%%%%%%%%%%%%%%%%%%%%%%%%%%%%%%
\subsection{Example: Methane}
\label{sec:dimmethane}

In order to illustrate the implementation of the DIM for molecules, 
we considered CH$_4$, which is a highly symmetric 
molecule. Furthermore, the spherical symmetry of the methane molecule
allows the system to be successfully described, for high energies, 
with a angular averaged potential \cite{Granados2016}.
We computed the HF molecular orbitals and energies of CH$_4$ employing 
the UGBS basis sets of carbon and hydrogen. Previous works 
\cite{Rothenberg1971,Hariharan1972} have shown that accurate molecular 
orbitals for methane require the inclusion of further polarisation 
functions to represent the distortion of the atoms. However, to solely 
isolate the effects of the basis set, we computed the atomic oscillation 
profiles and the molecular orbitals on the same footing. 
The charges obtained by direct inversion are given in 
Fig.~\ref{fig:ch4zeff} with dashed lines. 
Since the molecular orbitals are given by LCAO of carbon and 
hydrogen, the oscillations of the inverted charges are a 
consequence of the finite basis set of these atoms. 
To remove the most critical oscillations,  first, we must 
determine the oscillation profiles produced by the atomic carbon basis set.
We use Eq.~(\ref{eq:oscillation-prof}) to determine the 
$p_{1s}^{\mbox{\scriptsize UGBS}}$, $p_{2s}^{\mbox{\scriptsize UGBS}}$ 
and $p_{2p}^{\mbox{\scriptsize UGBS}}$ profiles of carbon. 
Then, we remove the oscillations by 
subtracting the carbon $p_{nl}^{\mbox{\scriptsize UGBS}}$ profiles 
to the corresponding inverted charges 
$Z_{i}^{\mbox{\scriptsize UGBS}}$ of CH$_4$.
The oscillations are removed for all orbitals except for the 
$2a_2$, which presents small oscillatory residues from the 
hydrogen basis set. Since the residual fluctuations are minimum
and near the nucleus, we proceeded to implement the depuration 
scheme as described in Section~\ref{sec:DIM}. We define a new 
parametric DIM charge equation,
\begin{eqnarray}
 Z(r) = \sum_j Z_j e^{-\alpha_j r} 
 + Z_{\mbox{\scriptsize H}} e^{-(\ln r - \ln \beta)^2/(2\gamma)} 
 + 1\,.
 \label{eq:moleculezDIM}
\end{eqnarray}
In contrast to the approximation proposed for 
atoms~(\ref{eq:atomiczDIM}), a second term has been added to the 
formula to account for the presence of the hydrogens.
The optimised parameters for the methane molecule are given in 
Table~\ref{tab:ch4parameters}, and the corresponding DIM 
charges are shown in Fig.~\ref{fig:ch4zeff}, with solid lines. 
The orbital energies obtained with these charges are also given in 
the table.

\begin{figure}[t]
\centering
\includegraphics[height=0.23\textheight]{figures/dimpot/zeff_1sCH4.eps}
\hspace{0.3cm}
\includegraphics[height=0.23\textheight]{figures/dimpot/zeff_2sCH4.eps}

\vspace{0.25cm}
\includegraphics[height=0.23\textheight]{figures/dimpot/zeff_2pCH4.eps}
\caption{Effective charges of CH$_4$; 
direct inversion (dashed line) and depurated inverted (solid line).}
\label{fig:ch4zeff}
\end{figure}

\begin{table}[H]
\centering
\begin{tabular}{cc|ccccc}
\hline
   $nl$ & $E$ &$Z$ & $\alpha$ & $\beta$ & $\gamma$ \\
\hline
   $1a_1$ & -11.1949  & 1.925280 & 0.641982 & & \\
          & & 0.953120 & 5.571510 & & \\
          & & 2.121600 & 1.500440 & & \\
   $2a_2$ & -0.9204 & 2.912200 & 3.149990 & & \\
          & & 2.087800 & 0.771371 & & \\
          & & 1.23640  &          & 2.329570 & 0.053420 \\
   $2t_1$ & -0.5042 & 0.901953 & 2.895140 & & \\
          & & 1.112030 & 0.388649 & & \\
          & & 2.986017 & 2.931210 & & \\
          & & 1.30182  &          & 2.169850 & 0.012616 \\
\hline
\end{tabular}
\caption{Energies and fitting parameters for the DIM effective charges 
(Eq.~(\ref{eq:moleculezDIM})), for CH$_4$.}
\label{tab:ch4parameters}
\end{table}


%%%%%%%%%%%%%%%%%%%%%%%%%%%%%%%%%%%%%%%%%%%%%%%%%%%%%%%%%%%%%%%%%%%%%
\subsection{Collisional Processes}

The orientation of the molecular targets is crucial for determining
the cross sections of collisional processes. However, the molecular 
orientations in most collisional experiments are generally not 
pre--established. Thus, the spherically averaged description of the system, 
assumed by the DIM potential is a valid approximation. In the 
following, we examine two collisional processes in the first--order 
approximation: proton--impact ionisation 
and single photoionisation. 

\subsubsection{Proton--Impact Ionisation}

Results for the proton--impact ionisation cross section for CH$_4$,
calculated under the first Born approximation, are given in 
Fig.~\ref{fig:ionch4}. 
Experimental data from Ref.~\cite{Rudd1983,Rudd1985} are displayed 
with symbols. The initial bound and the final continuum states 
of the molecule needed for the T--matrix computation 
(Eq.~(\ref{eq:transition-matrix})) were calculated 
with the DIM potentials from Section~\ref{sec:dimmethane}. 
The photoionisation cross section for high and intermediate energies
shows good agreement with the experimental results. The failure at
low energies is attributed to the validity of the first Born 
approximation and not to our DIM approach.

\begin{figure}[H]
\centering
\includegraphics[height=0.23\textheight]{figures/dimpot/born_ionch4.eps}
\caption{Proton--impact ionisation cross section for CH$_4$. Solid 
line: first--order DIM theoretical calculations. Symbols: experiments 
from Ref.~\cite{Rudd1983,Rudd1985}.}
\label{fig:ionch4}
\end{figure}

\subsubsection{Photoionisation}
\label{subsubsec:photoch4}

The photoionisation cross section for CH$_4$, calculated with the DIM 
potentials in a first order approximation, is shown in 
Fig.~\ref{fig:photoch4} (solid lines).
Good agreement with the experimental results (symbols) is found for 
high energy values and at the thresholds. The curve between $\sim$15 and
$\sim$300 eV shows the photoionisation from the outer $n=2$ shell, while 
the discontinuity at 300 eV corresponds to the threshold of
the $1a_1$ inner shell orbital. For low and intermediate photon--energies,
the agreement between our calculations and the experimental values 
from Ref.~\cite{Lukirskii1964,Henke1982,Samson1989} is not perfect.
Phenomena such as molecular orbital relaxation, possible 
collective contributions and correlation effects must be considered 
in further calculations. On the other hand, for the $1a_1$ inner shell
photoionisation, these effects are not significant, and we obtain a 
perfect agreement with the experimental results.
\begin{figure}[H]
\centering
\includegraphics[height=0.23\textheight]{figures/dimpot/photoch4.eps}
\caption{Single photoionisation cross section of CH$_4$.
Solid line: first--order DIM theoretical calculations. Symbols: experiments
from Ref.~\cite{Lukirskii1964,Henke1982,Samson1989}.}
\label{fig:photoch4}
\end{figure}


\section{Concluding remarks}

In this work, we explored the possibility of using pseudopotentials 
within the single electron model to calculate inelastic transitions. 
The first Born approximation was used to calculate proton--impact 
excitation, ionisation, electron capture and photoionisation. Two simple
atoms were studied, having a single electron in the outer shell. For 
hydrogen, we found excellent agreement for all the collisional processes, for 
low and intermediate energies. In the case of lithium, the only process
that can be calculated with reasonable accuracy is the proton--impact
excitation. We concluded that the range of validity is restrained to minimal
momentum transfers. 
The Depurated Inversion method, on the other hand, accurately reproduce
photoionisation experimental results for many--electron atoms. 

We extended the DIM for molecular systems. In this case, the inversion
procedure raises huge oscillation due to the finite size of the basis sets
involved 
in the Hartree--Fock orbital calculations. An additional step is included 
during the depuration scheme. In order to determine the oscillation profile
for a particular basis set, we computed the inverted atomic charges also in
a finite--differences framework. By subtracting the charges, it is possible
to isolate the oscillations corresponding to this particular basis set.
We used the DIM method to determine the effective potentials for CH$_4$.
These potentials are implemented in first--order proton--impact ionisation
and photoionisation cross sections calculations. For both processes, we
found good agreement with the experimental results. The main
discrepancies can be attributed to the fact that only first--order is 
considered in the perturbation theory.


\section*{Acknowledgement}
The authors thank the Consejo Nacional de Investigaciones 
Cient\'ificas y T\'ecnicas (CONICET), Universidad de Buenos Aires (UBA)
and Agencia Nacional de Promoci\'on Cient\'ifica y Tecnol\'ogica (ANPCyT)
for the grants that supported this work.


%%%%%%%%%%%%%%%%%%%%%%%%%%%%%%%%%%%%%%%%%%%%%%%%%%%%%%%%%%%%%%%%%%%%%
% \bibliography{dimmol}
% \bibliographystyle{achemso}

\begin{thebibliography}{10}

\bibitem{parsecwebpage}
PARSEC Home Page. 
\url{https://parsec.ices.utexas.edu/styled-2/},
(accessed Jan 18, 2019).

\bibitem{Chelikowsky1994}
Chelikowsky, J. R.; Troullier, N.; Saad, Y.
Finite--difference--pseudopotential method: Electronic structure calculations without a basis.
{\it Phys. Rev. Lett.} {\bf 1994}, 72, 1240--1243.

\bibitem{Mendez2015}
Mendez, A. M. P.
M\'etodo de Inversi\'on Depurada para Potenciales Locales en \'Atomos y Mol\'eculas.
Tesis de Licenciatura, Universidad Nacional de Salta, April 2015.

\bibitem{Mendez2016}
Mendez, A. M. P.; Mitnik, D. M.; Miraglia, J. E.
Depurated inversion method for orbital--specific exchange potentials.
{\it Int. J. Quantum Chem.} {\bf 2016}, 116, 1882--1890.

\bibitem{Mendez2018}
Mendez, A. M. P.; Mitnik, D. M.; Miraglia, J. E. 
Local Effective Hartree--Fock Potentials Obtained by the Depurated Inversion Method. 
In {\it Nov. Electron. Struct. Theory Gen. Innov. Strongly Correl. Syst.}; 
Hoggan, P. E., Ed.; 
Advances in Quantum Chemistry;
Academic Press, 2018;
Vol.~76; pp~117--132.

\bibitem{BatesBook}
Bates, D. R.; 
Theoretical Treatment of Collisions between Atomic Systems.
In {\it At. Mol. Process.};
Bates, D. R., Ed;
Pure and Applied Physics;
Elsevier, 1962;
Vol.~13, pp 549--621.

\bibitem{McDowell1961}
McDowell, M. R. C.; Peach, G.
Ionization of Lithium by Fast Protons and Electrons.
{\it Phys. Rev.} {\bf 1961}, 121, 1383--1387.

\bibitem{LeePindzola2011}
Lee, T.-G.; Pindzola, M. S.
Proton-impact excitation of lithium using a time--dependent 
close--coupling method.
{\it Phys. Rev. A} {\bf 2011}, 84, 052712.

\bibitem{Pindzola2016}
Pindzola, M. S.; Colgan, J.; Robicheaux, F.; Lee, T.-G.; Ciappina, M. F.;
Foster, M.; Ludlow, J. A.; Abdel-Naby,
 S. A.
Time-Dependent Close--Coupling Calculations for Ion--Impact Ionization 
of Atoms and Molecules. 
In {\it Advances In Atomic, Molecular, and Optical Physics};
Arimondo, E.; Lin, C. C.; Yelin,
 S. F., Ed,;
Academic Press,
 2016; Vol. 65,; pp 291--319.

\bibitem{Kirchner1998}
Kirchner, T.; Guly\'as, L.; L\"udde, H. J.; Engel, E.; Dreizler, R. M. 
Influence of electronic exchange on single and multiple processes in 
collisions between bare ions and noble--gas atoms.
{\it Phys. Rev. A} {\bf 1998}, 58, 2063--2076.

\bibitem{Fiori2001}
Fiori, M. R.; Jalbert, G.; Bielschowsky; C. E.; Cravero, W.
Ionization of lithium by impact of fast ions.
{\it Phys. Rev. A} {\bf 2001}, 64, 012705.

\bibitem{Szabo1996}
Szabo, A.; Ostlund, N. S.
{\it Modern Quantum Chemistry: Introduction to Advanced Electronic 
Structure Theory},
Dover Publications, Inc.: Mineola, New York, 1996.

\bibitem{Helgaker2000}
Helgaker, T.; J{\o}rgensen, P.; Olsen, J.
{\it Molecular Electronic-Structure Theory},
John Wiley {\&} Sons, Ltd: Chichester, UK, 2000.

\bibitem{Schaefer2004}
Schaefer, H. F. III
{\it Quantum Chemistry: The Development of Ab Initio Methods in
Molecular Electronic Structure Theory},
Dover Publications, Inc: Mineola, New York, 2004.

\bibitem{abinitwebpage}
ABINIT Home Page. 
\url{https://www.abinit.org/psp-tables},
(accessed Jan 18, 2019).

\bibitem{Hamann1979}
Hamann, D. R.; Schl\"uter, M.; Chiang, C.
Norm--Conserving Pseudopotentials.
{\it Phys. Rev. Lett.} {\bf 1979}, 43, 1494--1497.

\bibitem{FroeseFischer1997}
Froese Fischer, C.; Brage, T.; J\"onsson, P.
{\it Computational Atomic Structure: An MCHF Approach},
Institute of Physics Publishing: Bristol, UK, 1997.

\bibitem{Johnson2007}
Johnson, W. R. 
{\it Atomic Structure Theory: Lectures on Atomic Physics},
Springer--Verlag Berlin Heidelberg, 2007.

\bibitem{Brinkman1930}
Brinkman, H. C.; Kramers, H. A.
Zur Theorie der Einfangung von Elektronen durch $\alpha$--Teilchen.
{\it Proc. K. Akad. van Wet.} {\bf 1930}, 33, 973--984.

\bibitem{Henke1993}
Henke, B. L.; Gullikson, E. M.; Davis, J. C. 
X--Ray Interactions: Photoabsorption, Scattering, Transmission, and Reflection at $E$=50--30000 eV, $Z$=1--92.
{\it At. Data Nucl. Data Tables} {\bf 1993}, 54, 181--342.

\bibitem{Samson1990}
Samson, J. A. R.; Angel, G. C.
Single-- and double--photoionization cross sections of atomic nitrogen from threshold to 31 \AA.
{\it Phys. Rev. A} {\bf 1990}, 42, 1307--1312.

\bibitem{Samson2002}
Samson, J. A. R.; Stolte, W. C.
Precision measurements of the total photoionization cross--sections of He, Ne, Ar, Kr, and Xe.
{\it J. Electron Spectros. Relat. Phenomena} {\bf 2002}, 123, 265--276.

\bibitem{Stolte2016}
Stolte, W. C.; Jonauskas, V.; Lindle, D. W.; Sant'Anna, M. M.; Savin, D. W. 
Inner--shell Photoionization studies of neutral atomic nitrogen.
{\it Astrophys. J.} {\bf 2016}, 818, 149.

\bibitem{Ederer1964}
Ederer, D. L. 
Photoionization of the $4d$ electrons in Xenon.
{\it Phys. Rev. Lett.} {\bf 1964}, 13, 760--762.

\bibitem{Schipper1997}
Schipper, P. R. T.; Gritsenko, O. V.; Baerends, E. J. 
Kohn-Sham potentials corresponding to Slater and Gaussian basis set densities.
{\it Theor. Chem. Accounts: Theory, Comput. Model. (Theoretica Chim. Acta)} {\bf 1997}, 98, 16--24.

\bibitem{Mura1997}
Mura, M. E.; Knowles, P. J.; Reynolds, C. A.
Accurate numerical determination of Kohn--Sham potentials from electronic densities: I. Two--electron systems.
{\it J. Chem. Phys.} {\bf 1997}, 106, 9659--9667.

\bibitem{Jacob2011}
Jacob, C. R. 
Unambiguous optimization of effective potentials in finite basis sets.
{\it J. Chem. Phys.} {\bf 2011}, 135, 244102.

\bibitem{Gaiduk2013}
Gaiduk, A. P.; Ryabinkin, I. G.; Staroverov, V. N.
Removal of Basis--Set Artifacts in Kohn--Sham Potentials Recovered from Electron Densities.
{\it J. Chem. Theory Comput.} {\bf 2013}, 9, 3959--3964.

\bibitem{Schmidt1993}
Schmidt, M. W.; Baldridge, K. K.; Boatz, J. A.; Elbert, S. T.; Gordon, M. S.; Jensen, J. H.; Koseki, S.;
Matsunaga, N.; Nguyen, K. A.; Su, S.; Windus, T. L.; Dupuis, M.; Montgomery, J. A.
General atomic and molecular electronic structure system.
{\it J. Comput. Chem.} {\bf 1993}, 14, 1347--1363.

\bibitem{Gordon2005}
Gordon, M. S.; Schmidt, M. W.
Advances in electronic structure theory: GAMESS a decade later. 
In {\it Theory Appl. Comput. Chem.}; 
Dykstra, C. E.; Frenking, G.; Kim, K. S.; Scuseria, G. E. Eds;
Elsevier: Amsterdam, 2005; pp 1167--1189.

\bibitem{Granados2016}
Granados--Castro, C. M.
Application of Generalized Sturmian Basis Functions to Molecular Systems.
Ph.D. Thesis, Université de Lorraine, Metz, France and 
Universidad Nacional del Sur, Bahía Blanca, Argentina, 2016.

\bibitem{Rothenberg1971}
Rothenberg, S.; Schaefer, H. F.
Methane as a Numerical Experiment for Polarization Basis Function Selection.
{\it J. Chem. Phys.} {\bf 1971}, 54, 2764--2766.

\bibitem{Hariharan1972}
Hariharan, P. C.; Pople, J. A.
The effect of d-functions on molecular orbital energies for hydrocarbons.
{\it Chem. Phys. Lett.} {\bf 1972}, 16, 217--219.

\bibitem{Rudd1983}
Rudd, M. E.; DuBois, R. D.; Toburen, L. H.; Ratcliffe, C. A.; Goffe, T. V.
Cross sections for ionization of gases by 5--4000 keV protons and for electron capture by 5--150 keV protons.
{\it Phys. Rev. A} {\bf 1983}, 28, 3244--3257.

\bibitem{Rudd1985}
Rudd, M. E.; Kim, Y. K.; Madison, D. H.; Gallagher, J. W.
Electron production in proton collisions: total cross sections.
{\it Rev. Mod. Phys.} {\bf 1985}, 57, 965--994.

\bibitem{Lukirskii1964}
Lukirskii, A. P.; Brytov, I. A.; Zimkina, T. M.
{\it Optika i spektr.} {\bf 1964}, 17, 234.

\bibitem{Henke1982}
Henke, B. L.; Lee, P.; Tanaka, T. J.; Shimabukuro, R. L.; Fujikawa, B. K.
Low--energy X--ray interaction coefficients: Photoabsorption, scattering, and reflection: $E$=100--2000 eV $Z$=1--94.
{\it At. Data Nucl. Data Tables} {\bf 1982}, 27, 1--144.

\bibitem{Samson1989}
Samson, J. A. R.; Haddad, G. N.; Masuoka, T.; Pareek, P. N.; Kilcoyne, D. A. L.
Ionization yields, total absorption, and dissociative photoionization cross sections of CH4 from 110 to 950 \AA.
{\it J. Chem. Phys.} {\bf 1989}, 90, 6925--6932.

\end{thebibliography}

\end{document}
