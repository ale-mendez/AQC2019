\documentclass[10pt]{article}
\usepackage{amsmath,amssymb}
\usepackage{graphicx}
\usepackage{authblk}
\usepackage{float}
\usepackage{xcolor}
\usepackage[margin=1.1in]{geometry}


%%%%%%%%%%%%%%%%%%%%%%%%%%%%%%%%%%%%%%%%%%%%%%%%%%%%%%%%%%%%%%%%%%%%%
%%%%%%%%%%%%%%%%%%%%%%%%%%%%%%%%%%%%%%%%%%%%%%%%%%%%%%%%%%%%%%%%%%%%%
\begin{document}

\noindent
{\bf Referee 1:}

\vspace{0.2cm}
The authors thank the referee for the suggestions and corrections 
provided.

\begin{enumerate}
 \item 
 {\tt I am not an expert on collision theory, but surely there
 exist other published \\ calculations of proton-impact
 cross-sections for H, Li, C, N, Ne, Ar, and small \\ molecules. 
 If so, they should be reviewed in the introduction. \\ If not,
 that fact should be stated clearly. The paper should not give 
 the \\ impression that these are the first ever ab initio
 calculations of proton-impact \\ cross-sections.} \\ 
 {\tt On page 10, the authors state "Many ab initio and 
 semi-empirical theoretical \\ approximations have been developed
 to this end over the last century". This \\ statement should be
 supported with a few citations.} \\
 
 \vspace{-0.25cm}
 The references regarding published calculations of inelastic 
 processes and electronic structure methods were included as per 
 suggestion.
 
 \item {\tt It should be explained why the T-matrix of Eq.~(12) 
 involves orbitals rather than \\  N-electron wave-functions. Is 
 this an approximation? If so, what is its \\ justification?}
 
 The T--matrix of Eq.~(12) involves orbitals rather than 
 N--electron wave--functions because we are calculating one--electron
 transitions in a first order approximation, which involves only the
 initial and final orbitals of the target. The approximations
 employed in our calculations are now clearly stated in 
 Section 1. 
 
 \item {\tt Which Ar pseudopotential is shown in Fig.~1? A label is 
 missing.}
 
 The Ar pseudopotential is from PARSEC. We include the source in 
 the figure label. The dot-dash line label is missing 
 (corresponding to $-1/r$), but we are now referring to it in the
 text. 
 
 \item {\tt Axis labels in similar figures are not very consistent.
 \\ Why does Fig.~6 show differential cross-sections whereas 
 Fig. 8-9 show simply \\ cross-sections? \\ Why is the horizontal
 axis in Figs.~5 and 7 labeled "Proton Velocity" whereas in \\
 Figs.~6 (for another proton collision) it says "Energy"? \\
 Shouldn't the horizontal axis labels in Figs. 8, 9, 13 say 
 "Photon Energy" rather than simply "Energy"?}
 
 The inconsistency of the axis labels in the figures has been 
 fixed. Fig.~6 shows a single--differential cross section in order 
 to inspect the scattering of the ionised electron. By computing 
 $d\sigma/d\varepsilon_{\!f}$ we can inspect the behaviour
 of the form factor, which depends on the momentum transfer 
 vector. In Fig.~6, the $x$--axis corresponds to the scattered
 electron energy and not incident proton velocity, which is fixed
 to 1 a.u.). We clarified this in the figure and the text. The
 ``Photon Energy'' label was modified in the corresponding Figs.
 
 \item {\tt On page 11, the authors state "we considered the
 methane molecule, which has a \\ suitable spherical geometry". 
 It is a bad choice of words to call the geometry of CH4
 "spherical".}
 
 The referee is right; we rephased this sentence correctly.
 
 \item {\tt Did the basis set used for molecular calculations in
 Sec. 4.2 (UGBS) include \\ polarization functions? The UGBS as 
 originally defined has no polarization \\ functions and hence is 
 unsuitable for calculations on molecules.}
 
 The referee is correct. The molecular calculations 
 with UGBS should include polarisation functions (at least 
 $d$--functions). We are aware that including these functions increases the accuracy of the molecular energies, as is shown in Refs. [32,33].
 However, our interest is to understand the effect of the basis set,
 isolating them, and compute the atomic oscillations profiles as a way 
 to remedy the big fluctuations in the inverted charges.
 
 \item {\tt The analytic form of the second term of Eq. (31) should
 be justified.}
 
 \item {\tt Problems with Figs.~1-3 and 5-8: the line samples in
 the legends are too short \\ to distinguish between solid and
 dashed lines. The samples should be at least \\ 2.5 times longer.
 Also, instead of two dashed curves in the plots, one curve \\
 should be dashed and the other dot-and-dash to make it possible to
 distinguish \\ the curves when printed on a b/w printer.}
 
 The legend line length and line type have been modified as 
 suggested.
 
 \item {\tt Misprints.}
 
 The misprints were corrected. 
 
 \item {\tt On page 10, the wording "do to their generally
 multicentered and highly \\ non-central nature" is awkward.
 Suggest "due to their nonspherical symmetry and \\ multicenter
 character".}
 
 We thank the referee for her/his suggestion. We have modified it
 accordingly.
 
 \item {\tt Do not capitalize "depurated inversion method" when it
 is not abbreviated.}
 
 The capitalization has been corrected.

 \item {\tt The references should be formatted in the ACS style, as 
 per AQC instructions.}
 
 All references are now formatted in the ACS style.
 
\end{enumerate}

\end{document}
