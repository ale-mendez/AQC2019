\documentclass[10pt]{article}
\usepackage{amsmath,amssymb}
\usepackage{graphicx}
\usepackage{authblk}
\usepackage{float}
\usepackage{xcolor}
\usepackage[margin=1.1in]{geometry}


%%%%%%%%%%%%%%%%%%%%%%%%%%%%%%%%%%%%%%%%%%%%%%%%%%%%%%%%%%%%%%%%%%%%%
%%%%%%%%%%%%%%%%%%%%%%%%%%%%%%%%%%%%%%%%%%%%%%%%%%%%%%%%%%%%%%%%%%%%%
\begin{document}

\noindent
{\bf Referee 2:} 

\vspace{0.2cm}
The authors thank the referee for the suggestions and corrections provided.

\begin{enumerate}

 \item {\tt The description of numerical difficulties in obtaining 
 the pseudo-charges is \\ detailed, in particular for the 2s AO of the 
 lithium atom. One problem is that a \\ singularity appears due to charge 
 repulsion at around r=1 au. This resembles an \\ electron-electron cusp 
 and could be interpreted physically, because the charges \\ have been 
 defined (although an orbital may be empty) and could be averted by 
 \\ explicit correlation factors.}
 
 The referee is correct.
 
 \item {\tt Let me comment the CH4 application.
 
 This uses PPs to replace node-less AOs so it is not a case
 where the arguments \\ developed and illustrated by fid 1 and 2 are
 actually applied.
 
 In below (3), the claim concerning H-atoms is correct but 
 a physical \\ interpretation of the PP in this case is missing.}
 
 H has only one electron, and it does not seem to have any use to
 calculate the corresponding pseudopotential.
 However, there are many different proposals for PP, reproducing
 with high accuracy the main features of the wavefunctions, even
 for excited states.
 
 \item {\tt So, for CH4 oscillations in the AO basis are not modified
 by use of a PP. They come from defects in the GTO basis as the authors
 mention. }
 
 We did not make use of PPs in the CH4 calculation.
 
 \item {\tt A-why not use an ETO basis? The first term of eq (31) is 
 like an LCAO of Is ETOs.
 The n=2 shell could be compared with simple semi-cartesian Sturmians:

 S(1) = N (s+x+y+z) A(r) 

 S(2) = N (s-x-y+z) A(r) 
 
 S(3) = N (s-x+y-z) A(r) 

 S(4) = N (s+x-y-z) A(r) 

 Here N=1/2 for othonormal functions,  A(r) = exp(-($Z_{eff}$ r)/2 
 s= (2/a -r) and x,y,\\z are just the Cartesian functions. They are 
 bases for the Irreps a2 (S(1)) and \\ t1 (S(2), S(3), S(4)) 
 $Z_{eff}$ is to be optimised and could be given a value from the \\
 Clementi  tables.}
 
 Actually, in our first calculation for a CH4 potential, we employed 
 the SCF OCE MOs given by Moccia (1964), which uses Slaters. The
 inversion of such MOs showed incorrect behaviour  of the charge in the 
 origin and, therefore, they were not considered. However, it is worth
 inspecting the viability of employing ETOs in further calculations.

 \item {\tt It is well-known that modest GTO basis sets introduce 
 oscillations, since grad[$\rho$]/$\rho$ is not smooth.}
 
 This assertion is correct and it is clearly demonstrated in Ref. [28].
 
 \item {\tt The FD finite difference HF seems to cure this and it should 
 be benchmarked \\ against ETOs.}
 
 Yes, the FD method cure the oscillation issue, however, the poles due 
 to the nodes and asympotic divergences will still be present [5].

  \item {\tt I would suggest that NaH would be a nice example, with 
  a 10-electron core PP and \\ comparison to the detailed study of Li 
  and H.}
  
  This molecule suggestion will be taken into account for further 
  calculations.

 \item {\tt However, the desired properties are obtained quite well.
 
 A small point is that a PP is not the same as ‘GGA’ which is 
 a family of  functionals. The ABINIT GGA is PBE by default and the PP 
 is calculated within that framework. \\ The important point it that it 
 is ‘norm-conserving’.}
 
 The referee is right. 
 
 \item {\tt The English is rather poor.} 
 
 Misprints have been corrected, and awkward sentences have been
 rephrased.
 
 
\end{enumerate}

 
\end{document}
